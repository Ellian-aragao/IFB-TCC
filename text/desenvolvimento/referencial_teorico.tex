\chapter{Referencial teórico}

Como base do conteúdo a ser referenciado dentro deste trabalho, existem diversas tecnologias e conceitos que são de fundamental importância a compreensão de modo que seja entendido desde os comparativos a de fato como as ferramentas a serem tratadas funcionam e interagem com o kernel Linux. Diante disto, temos a seguir alguns conceitos a serem expostos de modo a clarificar a sua importância e a base das ferramentas a serem tratadas, nestas temos namespaces e cgroups como os conceitos mais importantes e utilizados diante de todos os softwares a serem comentados posteriormente.

\section{Namespace}
A tecnologia consiste em fazer o isolamento dos recursos a serem utilizados por determinado software que esteja rodando. Deste modo o namespace é uma camada de interação que se encontra entre a sua execução propriamente dita e sua interação com kernel, tal como um proxy \cite{kernelscheepers}.

Como comentado, esta camada que encontra-se entre o processo e o kernel, atua quando determinado software executa uma syscall (chamada de sistema) e as estruturas do processo, possuem os dados de determinado namespace, este que por sua vez delimita que recursos estão disponíveis para serem utilizados.

Desta forma cria-se uma virtualização dos recursos que são disponíveis e quais são eles, uma vez que a resposta para o software a ser executada é dada pelas configurações daquele namespace sob o qual a aplicação roda. Sendo esta a base para criação de containers, dos quais trata-se de um processo o qual roda de maneira isolada sob um sistema operacional.

Por fim, uma vez compreendida sua função, é necessário a compreensão de quais tipos de namespaces estão disponíveis uma vez que utilize-se o kernel posterior 5.6 do linux.

\section{Cgroups}
O controle de grupos, também chamado de cgroups, é a ferramenta para controle de uso do recurso, sendo esses de alguns tipos diferentes, mas em seu cerne trata-se de controle de quantidade de uso deste, ou seja quanto tempo ou banda de utilização de determinado recurso estaria liberado para um determinado processo do qual se encontra sob aquele cgroup.

Compreendido que se trata do uso de recurso, aqueles que estão sob controle de determinado grupo, pode-se limitar o uso da CPU e utilização de blocos de I/O (entrada e saída de dados). Deste modo pode-se determinar quanto tempo de processamento está disponível e/ou quantos bytes estão disponíveis na entrada e saída do processo, seja ela utilização de rede (internet), escrita e leitura de arquivos, etc...

\section{Containers}
falar sobre layer composition?

\section{Virtualização}
validar se está sendo analisada ferramenta de virtualização ou descartar tópico
