\chapter{Metodologia}

A segurança cibernética é de extrema importância na era digital, especialmente no que diz respeito à proteção de sistemas operacionais. Nesse contexto, o uso de técnicas de sandboxing tem se mostrado uma abordagem eficaz para mitigar ameaças e proteger sistemas GNU/Linux. Neste trabalho, realizaremos uma comparação detalhada de diversas ferramentas de sandboxing disponíveis nesses sistemas, utilizando uma metodologia que combina os tipos de pesquisa explicativa, além de uma abordagem quantitativa para análise dos dados.

\paragraph*{Pesquisa Explicativa} \mbox{}\\
Para a pesquisa explicativa, buscamos compreender os fundamentos e os princípios subjacentes a cada ferramenta de sandboxing. Exploramos a literatura técnica e acadêmica, bem como fontes confiáveis de informação, para obter uma compreensão mais profunda das técnicas de sandboxing empregadas, os mecanismos de isolamento e os modelos de segurança implementados em cada ferramenta.

%\paragraph*{Abordagem Quantitativa} \mbox{}\\
%Além da análise qualitativa, adotamos uma abordagem quantitativa para a comparação das ferramentas de sandboxing. Realizamos testes e medições objetivas para coletar dados quantitativos relacionados ao desempenho, consumo de recursos e eficácia na detecção de ameaças em diferentes cenários de teste. Utilizamos métricas apropriadas, como taxa de detecção, tempo de resposta e consumo de memória, para avaliar o desempenho de cada ferramenta.

\paragraph*{Comparação e Avaliação} \mbox{}\\
 Realizaremos uma análise comparativa das ferramentas selecionadas, levando em consideração critérios como facilidade de uso, flexibilidade, recursos de segurança, comunidade de suporte e integração com outros sistemas e ferramentas de segurança. Além disso, apresentamos os resultados quantitativos obtidos durante os testes, fornecendo uma visão objetiva das diferenças de desempenho entre as ferramentas.

\paragraph*{Cronograma} \mbox{} \\
Dado as metodologias tomadas nesta monografia, segue o cronograma das atividades a serem desenvolvidas:

\begin{center}
\begin{tabular}{|l|l|l|l|l|l|l|l|l|}
\hline
\multicolumn{1}{|c|}{Atividades/meses}   & \multicolumn{1}{c|}{5} & \multicolumn{1}{c|}{6} & \multicolumn{1}{c|}{7} & \multicolumn{1}{c|}{8} & \multicolumn{1}{c|}{9} & \multicolumn{1}{c|}{10} & \multicolumn{1}{c|}{11} & \multicolumn{1}{c|}{12} \\ \hline
Levantamento bibliográfico               & X                      & X                      & X                      &                        &                        &                         &                         &                         \\ \hline
Investigação dos critérios de comparação &                        & X                      & X                      &                        &                        &                         &                         &                         \\ \hline
Referencial Teórico                      &                        & X                      & X                      &                        &                        &                         &                         &                         \\ \hline
Escrita do PCC                           & X                      & X                      & X                      &                        &                        &                         &                         &                         \\ \hline
Detalhamento das ferramentas             &                        &                        & X                      & X                      & X                      &                         &                         &                         \\ \hline
Testes práticos                          &                        &                        & X                      & X                      & X                      & X                       &                         &                         \\ \hline
Análise dos resultados                   &                        &                        &                        &                        & X                      & X                       & X                       &                         \\ \hline
Escrita do TCC                           &                        &                        &                        & X                      & X                      & X                       & X                       & X                       \\ \hline
\end{tabular}
\end{center}

\paragraph*{Conclusão} \mbox{}\\
Ao final da entrega deste trabalho, esperamos fornecer uma análise abrangente das ferramentas de sandboxing em sistemas GNU/Linux, utilizando uma metodologia que engloba pesquisa explicativa, bem como uma abordagem quantitativa. Essa comparação detalhada permitirá que profissionais de segurança cibernética e administradores de sistemas façam uma escolha informada, selecionando a ferramenta mais adequada às suas necessidades e requisitos específicos.
