\chapter{Metodologia}

A segurança cibernética é de extrema importância na era digital, especialmente no que diz respeito à proteção de sistemas operacionais. Nesse contexto, o uso de técnicas de sandboxing tem se mostrado uma abordagem eficaz para mitigar ameaças e proteger sistemas GNU/Linux. Neste trabalho, realizaremos uma comparação detalhada de diversas ferramentas de sandboxing disponíveis nesses sistemas, utilizando uma metodologia que combina os tipos de pesquisa exploratória, descritiva e explicativa, além de uma abordagem quantitativa para análise dos dados.

\subsection{Pesquisa Exploratória}
Realizamos uma pesquisa exploratória para coletar informações preliminares sobre as ferramentas de sandboxing em sistemas GNU/Linux. Nessa etapa, utilizamos fontes como artigos científicos, documentação técnica e fóruns de discussão para compreender as características gerais das ferramentas, seus casos de uso comuns e as principais questões relacionadas à sua implementação.

\subsection{Pesquisa Descritiva}
Conduzimos uma pesquisa descritiva, na qual coletamos dados detalhados sobre as ferramentas selecionadas. Essa etapa envolveu a análise de documentação oficial, estudos de caso e relatórios de desempenho, buscando informações sobre funcionalidades específicas, requisitos de instalação, configuração e uso, além de métricas de desempenho e segurança.

\subsection{Pesquisa Explicativa} 
Para a pesquisa explicativa, buscamos compreender os fundamentos e os princípios subjacentes a cada ferramenta de sandboxing. Exploramos a literatura técnica e acadêmica, bem como fontes confiáveis de informação, para obter uma compreensão mais profunda das técnicas de sandboxing empregadas, os mecanismos de isolamento e os modelos de segurança implementados em cada ferramenta.

\subsection{Abordagem Quantitativa}
Além da análise qualitativa, adotamos uma abordagem quantitativa para a comparação das ferramentas de sandboxing. Realizamos testes e medições objetivas para coletar dados quantitativos relacionados ao desempenho, consumo de recursos e eficácia na detecção de ameaças em diferentes cenários de teste. Utilizamos métricas apropriadas, como taxa de detecção, tempo de resposta e consumo de memória, para avaliar o desempenho de cada ferramenta.

\subsection{Comparação e Avaliação}
 Realizaremos uma análise comparativa das ferramentas selecionadas, levando em consideração critérios como facilidade de uso, flexibilidade, recursos de segurança, comunidade de suporte e integração com outros sistemas e ferramentas de segurança. Além disso, apresentamos os resultados quantitativos obtidos durante os testes, fornecendo uma visão objetiva das diferenças de desempenho entre as ferramentas.

\subsection{Conclusão}
Ao final da entrega deste trabalho, esperamos fornecer uma análise abrangente das ferramentas de sandboxing em sistemas GNU/Linux, utilizando uma metodologia que engloba pesquisa exploratória, descritiva e explicativa, bem como uma abordagem quantitativa. Essa comparação detalhada permitirá que profissionais de segurança cibernética e administradores de sistemas façam uma escolha informada, selecionando a ferramenta mais adequada às suas necessidades e requisitos específicos.