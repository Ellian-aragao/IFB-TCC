\section{LXC}

O LXC (Linux Containers) é uma tecnologia de virtualização baseada em contêineres que permite a execução de múltiplos ambientes isolados em um único sistema Linux. Ele oferece uma abordagem leve e eficiente para virtualização, fornecendo uma alternativa às máquinas virtuais tradicionais \cite{lxc-docs}.

% \subsection{Gerenciar contêineres}
\paragraph*{Gerenciamento de contêineres}\mbox{}\\

Uma das principais características do LXC é a sua capacidade de criar e gerenciar contêineres Linux completos. Cada contêiner LXC possui seu próprio ambiente isolado, incluindo sistema operacional, bibliotecas, aplicativos e recursos alocados, mas compartilha o mesmo kernel do sistema operacional hospedeiro. Essa abordagem compartilhada resulta em um melhor desempenho e utilização de recursos em comparação com a virtualização completa \cite{lxc-docs, container-security}.

O LXC oferece um conjunto abrangente de ferramentas de linha de comando e APIs para criar e gerenciar contêineres. Ele permite que os administradores configurem e personalizem os recursos de cada contêiner, incluindo CPU, memória, armazenamento e rede. Isso proporciona um controle granular sobre a alocação de recursos e garante um compartilhamento justo entre os contêineres.

\paragraph*{Isolamento}\mbox{}\\
 Uso de tecnologias avançadas de isolamento, como cgroups e namespaces do kernel do Linux. Operando de forma independente e segura.


\paragraph*{Flexível e escalável}\mbox{}\\
 Pode ser usado em uma variedade de cenários, desde a execução de aplicativos individuais em contêineres isolados até a criação de ambientes de teste e desenvolvimento completos. Sua flexibilidade também o torna uma escolha popular para implantação em nuvem, permitindo a rápida criação e destruição de contêineres em escala.

\paragraph*{Integração}\mbox{}\\
  Como o Docker e o Kubernetes. O LXC pode ser usado como uma das opções de execução para contêineres Docker, fornecendo um ambiente isolado para a execução de aplicativos Docker. Além disso, o LXC pode ser usado como o mecanismo de execução subjacente para clusters Kubernetes, permitindo a orquestração e gerenciamento de contêineres em escala.