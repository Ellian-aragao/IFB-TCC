\section{AppArmor}
É um sistema de controle de acesso obrigatório (MAC - Mandatory Access Control) o qual oferece recursos de segurança adicionais ao SO GNU/Linux. Ele permite ao administrador do sistema criar políticas de segurança para controlar o acesso de processos aos recursos do SO, tais como arquivos, diretórios, portas de rede e outros \cite{apparmor-gitlab}.
 
\paragraph*{Configuração}\mbox{}\\

Sua configuração envolve a definição de perfis de segurança para cada aplicativo, estes especificam recursos que aplicativo pode acessar e quais ações pode realizar. Essa configuração pode ser feita por meio de arquivos de perfil ou usando ferramentas de linha de comando fornecidas pelo AppArmor \cite{apparmor-Ubuntu-Linux-server, apparmor-gitlab}.

Embora seja uma ferramenta poderosa, para aumentar a segurança do sistema, não é uma solução completa. Recomenda-se que os administradores de sistema adotem uma abordagem de camadas para a segurança, através da combinação do AppArmor com outras medidas de proteção, tais como atualizações regulares de segurança.

\paragraph*{Integração com Linux}\mbox{}\\

Uma das principais características é sua integração com o kernel Linux, o qual aproveita-se dos recursos de segurança já fornecidos pelo kernel, tais como namespaces e cgroups, para isolamento dos processos. Essa abordagem garante um bom desempenho e um gasto baixo de recursos adicionais em troca da segurança \cite{hardening-linux}.

\paragraph*{Gerência de perfis}\mbox{}\\

As políticas de segurança de forma granular provém seu gerenciamento uma boa flexibilidade através de perfis, representados por arquivos de configuração, definem permissões e restrições de acesso para cada processo em execução no SO.

O gerenciamento envolve três etapas principais: criação, configuração e aplicação dos perfis.

A criação de um perfil é a etapa em que um administrador define quais permissões e restrições são desejadas para um aplicativo específico. O perfil é descrito em uma linguagem de configuração, Profile Language (AAPL), ou usando uma abordagem baseada em aprendizado de perfil, onde o AppArmor registra as ações do aplicativo durante uma sessão de treinamento e cria um perfil com base nesses registros \cite{apparmor-gitlab}.

A configuração de um perfil envolve especificar as permissões necessárias para que o aplicativo funcione corretamente. Isso pode incluir permissões para acessar arquivos, diretórios, dispositivos, sockets de rede e outras entidades do sistema \cite{apparmor-security-and-integration}.

Quanto a aplicação, está é feita de forma simples através da sua interface de linha de comando, aliada a abordagem de aprendizagem, sua utilização tende a ser facilmente feita.

Além disso, o software possui suporte a herança de perfils, onde é possível criar perfis base e perfis derivados. Os perfis derivados herdam as permissões e restrições dos perfis base, permitindo a criação de hierarquias de perfis, facilitando o gerenciamento de políticas de segurança \cite{debian-handbook}.

Destacando também fornecimento de recursos avançados de auditoria e registro de eventos, dos quais logs registram violações de segurança, permitindo a analise de comportamentos inesperados ou não previamente mapeados.

\paragraph*{Exemplos}\mbox{}\\
 Considere um servidor web que executa um aplicativo da web. Podemos criar um perfil de segurança específico para esse aplicativo, limitando seu acesso apenas aos arquivos e diretórios necessários para operar corretamente. Isso restringe o impacto de um possível ataque, limitando as ações que um invasor pode realizar dentro do ambiente restrito do aplicativo.

Outro exemplo é a utilização em desktops, onde a execução de programas de cunho desconhecido podem ser executadas de forma segura evitando que um malware infiltre-se no sistema.
