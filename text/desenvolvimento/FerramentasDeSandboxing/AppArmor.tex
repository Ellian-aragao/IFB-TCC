\section{AppArmor}
O AppArmor é um sistema de controle de acesso obrigatório (MAC - Mandatory Access Control) que oferece recursos de segurança adicionais ao sistema operacional Linux. Ele permite aos administradores do sistema criar políticas de segurança personalizadas para controlar o acesso de processos e aplicativos a recursos do sistema, como arquivos, diretórios, portas de rede e outros.

O objetivo do AppArmor é restringir o acesso a esses recursos, limitando as ações que um processo pode realizar, a fim de evitar a execução de código malicioso ou não autorizado. Ao utilizar o AppArmor, os administradores podem criar perfis de segurança que definem as permissões e restrições para cada aplicativo em execução no sistema.

\subsection{Configuração}
A configuração do AppArmor envolve a definição de perfis de segurança para cada aplicativo. Esses perfis especificam quais recursos do sistema o aplicativo pode acessar e quais ações ele pode realizar. Essa configuração pode ser feita por meio de arquivos de perfil ou usando ferramentas de linha de comando fornecidas pelo AppArmor.

Embora o AppArmor seja uma ferramenta poderosa para aumentar a segurança do sistema, é importante notar que ele não é uma solução completa. É recomendado que os administradores de sistema adotem uma abordagem em camadas para a segurança, combinando o uso do AppArmor com outras medidas de proteção, como atualizações regulares de segurança e práticas de segurança recomendadas.

\subsection{Integração com linux}

Uma das principais características do AppArmor é sua integração com o kernel do Linux. Ele aproveita os recursos de segurança fornecidos pelo kernel, como namespaces e cgroups, para isolar os processos em seus próprios ambientes protegidos. Essa abordagem garante um bom desempenho e uma pegada de recursos mínima.

\subsection{gerenciando os perfis}

O AppArmor, no Linux, oferece um sistema flexível de gerenciamento de perfis que permite aos administradores do sistema controlar as políticas de segurança de forma granular. Os perfis são arquivos de configuração que definem as permissões e restrições de acesso para cada aplicativo ou processo em execução no sistema.

No AppArmor, o gerenciamento de perfis envolve três etapas principais: criação, configuração e aplicação dos perfis.

A criação de um perfil é a etapa em que um administrador define as permissões e restrições desejadas para um aplicativo específico. O perfil é geralmente escrito em uma linguagem de configuração, como o AppArmor Profile Language (AAPL) ou usando uma abordagem baseada em aprendizado de perfil, onde o AppArmor registra as ações do aplicativo durante uma sessão de treinamento e cria um perfil com base nesses registros.

A configuração de um perfil envolve especificar as permissões necessárias para que o aplicativo funcione corretamente. Isso pode incluir permissões para acessar arquivos, diretórios, dispositivos, sockets de rede e outras entidades do sistema. As restrições também podem ser aplicadas, limitando as ações que o aplicativo pode executar, como impedir o acesso a certas partes do sistema de arquivos ou restringir o uso de recursos de rede.

Uma vez que o perfil tenha sido criado e configurado, ele pode ser aplicado ao aplicativo ou processo específico. O AppArmor suporta diferentes métodos de aplicação de perfil, como associar o perfil a um binário específico, atribuí-lo a um namespace ou usar as ferramentas de linha de comando do AppArmor para aplicar o perfil dinamicamente.

Além disso, o AppArmor suporta o conceito de herança de perfil, onde é possível criar perfis base e perfis dependentes. Os perfis dependentes herdam as permissões e restrições dos perfis base, permitindo a criação de hierarquias de perfis que facilitam o gerenciamento de políticas de segurança em larga escala \cite{debian-handbook}.

É importante destacar que o AppArmor fornece recursos avançados de auditoria e registro de eventos. Os logs do AppArmor registram violações de segurança, permitindo aos administradores analisar e solucionar problemas relacionados à configuração de perfil e detectar possíveis ameaças. 

\subsection{exemplos}
 Considere um servidor web que executa um aplicativo da web. Podemos criar um perfil de segurança específico para esse aplicativo, limitando seu acesso apenas aos arquivos e diretórios necessários para operar corretamente. Isso restringe o impacto de um possível ataque, limitando as ações que um invasor pode realizar dentro do ambiente restrito do aplicativo.

Outro exemplo é o uso do AppArmor em ambientes de desktop, onde aplicativos como navegadores da web são comumente usados. Ao aplicar perfis de segurança a esses aplicativos, podemos restringir seu acesso a informações confidenciais, como arquivos pessoais do usuário, protegendo assim a privacidade e evitando a exfiltração de dados.