\section{Capsicum}

Ferramenta de segurança e sandboxing desenvolvida para o sistema operacional FreeBSD, embora também tenha sido portado para outros sistemas, como o Linux. Ele oferece recursos avançados de isolamento de processos e restrição de acesso, com foco na minimização de vulnerabilidades e na proteção de aplicativos contra ataques cibernéticos.

As principais características do Capsicum é a implementação do modelo de sandbox baseado em recursos, que permite aos desenvolvedores restringir o acesso de aplicativos somente aos recursos necessários para sua execução. Isso é alcançado através do conceito de "capabilidades" (capabilities), que são permissões granulares que um processo pode ter para acessar recursos do sistema. O Capsicum permite que os desenvolvedores reduzam essas capabilidades para um processo, limitando assim seu poder e evitando a exploração de vulnerabilidades.

\subsection{Separação de contexto}

Tendo a separação de contexto, que é alcançada através do uso de descritores de arquivos e recursos encapsulados chamados "sandbox". Essa separação de contexto permite que um processo tenha acesso somente a um subconjunto específico de recursos, limitando sua capacidade de interagir com outros processos ou modificar recursos não autorizados.

Suporta o uso de descritores de arquivos com limitação de escopo, onde um processo pode restringir o acesso a um conjunto específico de descritores de arquivos. Isso ajuda a prevenir a escalada de privilégios, impedindo que um processo acesse outros arquivos ou recursos além dos permitidos.

\subsection{exemplos}

Uma aplicação prática do Capsicum é a proteção de aplicativos com vários componentes que precisam interagir entre si, mas sem expor todos os recursos do sistema a cada componente. Com o Capsicum, é possível definir capabilidades específicas para cada componente, limitando assim seu acesso a recursos específicos e reduzindo o impacto caso algum componente seja comprometido.