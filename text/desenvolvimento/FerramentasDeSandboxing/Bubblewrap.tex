\section{Bubblewrap}

 Ferramenta de sandboxing amplamente utilizada no ecossistema Linux. Ele fornece um ambiente seguro e isolado para a execução de aplicativos, permitindo que eles operem com um nível mínimo de privilégios e restrições de acesso. O Bubblewrap baseia-se em recursos avançados do kernel do Linux, como namespaces e cgroups, para oferecer isolamento de processos e limitação de acesso.

\subsection{Isolamento de processos}
O isolamento de processos é uma das principais funcionalidades do Bubblewrap. Ele cria um ambiente separado para a execução de aplicativos, isolando-os uns dos outros e do sistema operacional hospedeiro. Cada aplicativo em execução dentro do sandbox do Bubblewrap é encapsulado em seu próprio conjunto de namespaces, incluindo namespaces de processo, rede, sistema de arquivos e hostname. Isso impede que os processos dentro do sandbox afetem outros processos ou acessem recursos do sistema não autorizados.


\subsection{Limitação de acesso}
A limitação de acesso é outra característica fundamental do Bubblewrap. Ele permite que os administradores configurem políticas de segurança detalhadas para restringir o acesso dos aplicativos a recursos específicos do sistema. Por exemplo, é possível limitar o acesso a arquivos, diretórios, dispositivos, sockets de rede e outros recursos essenciais. Essa abordagem granular ajuda a evitar que aplicativos comprometidos acessem dados sensíveis ou executem operações indesejadas.

 Sendo amplamente utilizado em diferentes cenários. É uma parte essencial do ecossistema de empacotamento de aplicativos Flatpak, permitindo que os aplicativos sejam executados em um ambiente seguro e isolado. É bastante utilizado em outras tecnologias de empacotamento, como o Snap, para garantir a segurança e o isolamento de aplicativos. Ele também pode ser usado para criar ambientes de teste isolados, facilitando a execução de aplicativos sem risco de impacto no sistema operacional hospedeiro.

\subsection{Exemplo}

Um exemplo prático do uso do Bubblewrap é a execução de um navegador da web em um sandbox. Ao executar o navegador dentro de um sandbox do Bubblewrap, é possível limitar seu acesso ao sistema de arquivos, restringir permissões de rede e impedir que ele acesse informações confidenciais. Isso aumenta a segurança, reduzindo a superfície de ataque em caso de exploração de vulnerabilidades do navegador.