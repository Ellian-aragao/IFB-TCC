\chapter{Ferramentas de Sandboxing}
\label{charpter:Ferramentas de Sandboxing}

Através das bases apresentadas no referencial teórico \ref{chp:referencial_teorico}, é possível fazer a construção de uma ferramenta de sandbox usando dos recursos falados, namespaces, cgroups, de forma manual. Entretanto, existem diversos softwares que fazem toda essa gestão destes recursos de baixo nível, ou seja próximos do sistema operacional, de forma amigável ao usuário que não tenha familiaridade aos conceitos referidos.

Diante disso a seguir serão apresentadas as ferramentas selecionadas e algumas informações sobre cada ferramenta. 

\section{AppArmor}
O AppArmor é um sistema de controle de acesso obrigatório (MAC - Mandatory Access Control) que oferece recursos de segurança adicionais ao sistema operacional Linux. Ele permite aos administradores do sistema criar políticas de segurança personalizadas para controlar o acesso de processos e aplicativos a recursos do sistema, como arquivos, diretórios, portas de rede e outros.

O objetivo do AppArmor é restringir o acesso a esses recursos, limitando as ações que um processo pode realizar, a fim de evitar a execução de código malicioso ou não autorizado. Ao utilizar o AppArmor, os administradores podem criar perfis de segurança que definem as permissões e restrições para cada aplicativo em execução no sistema.

\subsection{Configuração}
A configuração do AppArmor envolve a definição de perfis de segurança para cada aplicativo. Esses perfis especificam quais recursos do sistema o aplicativo pode acessar e quais ações ele pode realizar. Essa configuração pode ser feita por meio de arquivos de perfil ou usando ferramentas de linha de comando fornecidas pelo AppArmor.

Embora o AppArmor seja uma ferramenta poderosa para aumentar a segurança do sistema, é importante notar que ele não é uma solução completa. É recomendado que os administradores de sistema adotem uma abordagem em camadas para a segurança, combinando o uso do AppArmor com outras medidas de proteção, como atualizações regulares de segurança e práticas de segurança recomendadas.

\subsection{Integração com linux}

Uma das principais características do AppArmor é sua integração com o kernel do Linux. Ele aproveita os recursos de segurança fornecidos pelo kernel, como namespaces e cgroups, para isolar os processos em seus próprios ambientes protegidos. Essa abordagem garante um bom desempenho e uma pegada de recursos mínima.

\subsection{gerenciando os perfis}

O AppArmor, no Linux, oferece um sistema flexível de gerenciamento de perfis que permite aos administradores do sistema controlar as políticas de segurança de forma granular. Os perfis são arquivos de configuração que definem as permissões e restrições de acesso para cada aplicativo ou processo em execução no sistema.

No AppArmor, o gerenciamento de perfis envolve três etapas principais: criação, configuração e aplicação dos perfis.

A criação de um perfil é a etapa em que um administrador define as permissões e restrições desejadas para um aplicativo específico. O perfil é geralmente escrito em uma linguagem de configuração, como o AppArmor Profile Language (AAPL) ou usando uma abordagem baseada em aprendizado de perfil, onde o AppArmor registra as ações do aplicativo durante uma sessão de treinamento e cria um perfil com base nesses registros.

A configuração de um perfil envolve especificar as permissões necessárias para que o aplicativo funcione corretamente. Isso pode incluir permissões para acessar arquivos, diretórios, dispositivos, sockets de rede e outras entidades do sistema. As restrições também podem ser aplicadas, limitando as ações que o aplicativo pode executar, como impedir o acesso a certas partes do sistema de arquivos ou restringir o uso de recursos de rede.

Uma vez que o perfil tenha sido criado e configurado, ele pode ser aplicado ao aplicativo ou processo específico. O AppArmor suporta diferentes métodos de aplicação de perfil, como associar o perfil a um binário específico, atribuí-lo a um namespace ou usar as ferramentas de linha de comando do AppArmor para aplicar o perfil dinamicamente.

Além disso, o AppArmor suporta o conceito de herança de perfil, onde é possível criar perfis base e perfis dependentes. Os perfis dependentes herdam as permissões e restrições dos perfis base, permitindo a criação de hierarquias de perfis que facilitam o gerenciamento de políticas de segurança em larga escala \cite{debian-handbook}.

É importante destacar que o AppArmor fornece recursos avançados de auditoria e registro de eventos. Os logs do AppArmor registram violações de segurança, permitindo aos administradores analisar e solucionar problemas relacionados à configuração de perfil e detectar possíveis ameaças. 

\subsection{exemplos}
 Considere um servidor web que executa um aplicativo da web. Podemos criar um perfil de segurança específico para esse aplicativo, limitando seu acesso apenas aos arquivos e diretórios necessários para operar corretamente. Isso restringe o impacto de um possível ataque, limitando as ações que um invasor pode realizar dentro do ambiente restrito do aplicativo.

Outro exemplo é o uso do AppArmor em ambientes de desktop, onde aplicativos como navegadores da web são comumente usados. Ao aplicar perfis de segurança a esses aplicativos, podemos restringir seu acesso a informações confidenciais, como arquivos pessoais do usuário, protegendo assim a privacidade e evitando a exfiltração de dados.
\section{Bubblewrap}

 Ferramenta de sandboxing amplamente utilizada no ecossistema Linux, como no seu derivado flatpak, Ele fornece um ambiente seguro e isolado para a execução de aplicativos, permitindo que eles operem com um nível mínimo de privilégios e restrições de acesso. O Bubblewrap baseia-se em recursos avançados do kernel do Linux, como namespaces e cgroups, para oferecer isolamento de processos e limitação de acessos \cite{bubblewrap-github, bubblewrap-archlinux}.

\paragraph*{Isolamento de processos}\mbox{}\\
Ele cria um ambiente separado para a execução de aplicativos, isolando-os uns dos outros e do sistema operacional hospedeiro. Cada aplicativo em execução dentro do sandbox do Bubblewrap é encapsulado em seu próprio conjunto de namespaces, como de rede, sistema de arquivos e hostname. Isso impede que os processos dentro do sandbox afetem outros processos ou acessem recursos do sistema não autorizados \cite{linux-containers-virtualization}.

\paragraph*{Limitação de acesso}\mbox{}\\
A limitação de acesso é outra característica fundamental do Bubblewrap. Ele permite que os administradores configurem políticas de segurança detalhadas para restringir o acesso dos aplicativos a recursos específicos do sistema. Por exemplo, é possível limitar o acesso a arquivos, diretórios, dispositivos, sockets de rede e outros recursos essenciais. Essa abordagem granular ajuda a evitar que aplicativos comprometidos acessem dados sensíveis ou executem operações indesejadas \cite{linux-security-redhat, bubblewrap-archlinux}.

 Sendo amplamente utilizado em diferentes cenários, é uma parte essencial do ecossistema de empacotamento de aplicativos Flatpak, permitindo que os aplicativos sejam executados em um ambiente seguro e isolado. Ele também pode ser usado para criar ambientes de teste isolados, facilitando a execução de aplicativos sem risco de impacto no sistema operacional hospedeiro 
 \cite{bubblewrap-github, bubblewrap-archlinux}.

\paragraph*{Exemplo}\mbox{}\\
Um exemplo prático do uso é a execução de um navegador da web em um sandbox. Ao executar o navegador dentro de um sandbox do Bubblewrap, é possível limitar seu acesso ao sistema de arquivos, restringir permissões de rede e impedir que ele acesse informações confidenciais. Isso aumenta a segurança, reduzindo a superfície de ataque em caso de exploração de vulnerabilidades do navegador.
\section{Capsicum}

Ferramenta de segurança e sandboxing desenvolvida para o sistema operacional FreeBSD, embora também tenha sido portado para outros sistemas, como o Linux. Ele oferece recursos avançados de isolamento de processos e restrição de acesso, com foco na minimização de vulnerabilidades e na proteção de aplicativos contra ataques cibernéticos.

As principais características do Capsicum é a implementação do modelo de sandbox baseado em recursos, que permite aos desenvolvedores restringir o acesso de aplicativos somente aos recursos necessários para sua execução. Isso é alcançado através do conceito de "capabilidades" (capabilities), que são permissões granulares que um processo pode ter para acessar recursos do sistema. O Capsicum permite que os desenvolvedores reduzam essas capabilidades para um processo, limitando assim seu poder e evitando a exploração de vulnerabilidades.

\subsection{Separação de contexto}

Tendo a separação de contexto, que é alcançada através do uso de descritores de arquivos e recursos encapsulados chamados "sandbox". Essa separação de contexto permite que um processo tenha acesso somente a um subconjunto específico de recursos, limitando sua capacidade de interagir com outros processos ou modificar recursos não autorizados.

Suporta o uso de descritores de arquivos com limitação de escopo, onde um processo pode restringir o acesso a um conjunto específico de descritores de arquivos. Isso ajuda a prevenir a escalada de privilégios, impedindo que um processo acesse outros arquivos ou recursos além dos permitidos.

\subsection{exemplos}

Uma aplicação prática do Capsicum é a proteção de aplicativos com vários componentes que precisam interagir entre si, mas sem expor todos os recursos do sistema a cada componente. Com o Capsicum, é possível definir capabilidades específicas para cada componente, limitando assim seu acesso a recursos específicos e reduzindo o impacto caso algum componente seja comprometido.
\section{Docker}

O Docker é uma plataforma de software que permite aos desenvolvedores criar, implantar e executar aplicativos em contêineres. O principal objetivo do Docker é facilitar a implantação e o gerenciamento de aplicativos, fornecendo um ambiente consistente e isolado para sua execução \cite{docker-docs}.

\paragraph*{Containers}\mbox{}\\
 Um container é uma unidade de software que encapsula um aplicativo e todas as suas dependências, incluindo bibliotecas, frameworks e arquivos de configuração. Diferentemente das máquinas virtuais tradicionais, os contêineres compartilham o mesmo kernel do sistema operacional hospedeiro, o que os torna mais leves e eficientes \cite{linux-containers-virtualization,docker-docs}.

\paragraph*{Portabilidade}\mbox{}\\
A portabilidade é outra característica-chave do Docker. Os contêineres Docker são independentes da plataforma e podem ser executados em qualquer ambiente que tenha o Docker instalado, seja localmente em um computador de desenvolvimento ou em um ambiente de nuvem como o AWS, Google Cloud Platform ou Microsoft Azure. Isso facilita a implantação consistente de aplicativos em diferentes ambientes, eliminando problemas de compatibilidade \cite{docker-book,docker-aws}.

\paragraph*{Gerenciamento de imagens}\mbox{}\\

Uma imagem Docker é uma representação estática de um contêiner, contendo todas as dependências e configurações necessárias. As imagens são criadas usando arquivos de configuração chamados Dockerfiles, que descrevem os passos para configurar o ambiente do contêiner. Isso torna a criação e o compartilhamento de aplicativos e suas dependências mais fáceis e reproduzíveis.

\paragraph*{Escabilidade}\mbox{}\\

Com o Docker Swarm ou o Kubernetes, é possível implantar aplicativos em clusters de máquinas, permitindo a execução de várias réplicas de um aplicativo para lidar com altas cargas de tráfego. Essa capacidade de escalabilidade horizontal ajuda a garantir a disponibilidade e o desempenho de aplicativos em ambientes de produção \cite{docker-docs,docker-book,docker-aws}.

Permitindo a criação de redes virtuais isoladas para os contêineres. Isso possibilita a comunicação segura entre contêineres, bem como a exposição controlada de portas para comunicação com o mundo externo.
\section{Firejail}

Uma ferramenta de sandboxing que oferece um ambiente seguro para a execução de aplicativos Linux. Sua principal finalidade é isolar processos e restringir o acesso a recursos do sistema, fornecendo uma camada adicional de segurança.

Uma das características mais importantes do Firejail é o uso de namespaces do kernel do Linux para criar um ambiente isolado para o aplicativo. Ele utiliza namespaces como PID, rede, sistema de arquivos e usuários para separar o aplicativo em execução do restante do sistema. Isso evita que o aplicativo acesse ou modifique recursos sensíveis do sistema, reduzindo o risco de ataques \cite{firejail-archwiki}.

Ele também fornece controle granular sobre as permissões de acesso do aplicativo. Por meio de perfis de segurança, podendo especificar quais recursos o aplicativo pode acessar e quais ações ele pode realizar. Isso inclui restrições em acesso a arquivos, diretórios, sockets de rede, dispositivos e muito mais. Com essas permissões definidas, o Firejail garante que o aplicativo funcione somente dentro dos limites especificados.

\paragraph*{Segurança}\mbox{}\\
 Como a proteção contra a fuga de informações através de áreas de transferência (clipboard), prevenção de captura de tela e restringir o acesso a recursos de servidores de exibição gráfica, tais como Xorg (X11) e Wayland. Essas medidas extras ajudam a proteger informações sensíveis e impedir que aplicativos maliciosos realizem ações indesejadas.
 
Ele pode ser executado diretamente na linha de comando, envolvendo o comando do aplicativo que se deseja executar em sandbox. Também é possível criar perfis de segurança personalizados que podem ser compartilhados e reutilizados nas aplicações.

\paragraph*{}\mbox{Utilização}\\
O Firejail é frequentemente utilizado para a execução de navegadores da web, clientes de email e outros aplicativos suscetíveis a ataques. Com o Firejail, é possível mitigar o impacto de vulnerabilidades do aplicativo, limitando sua capacidade de acessar recursos do sistema e restringindo sua interação com outros processos em execução.
\section{LXC}
\section{OpenVZ}

O OpenVZ é uma plataforma de virtualização de sistema operacional que permite a criação de ambientes virtuais isolados, chamados de "contêineres", em um único servidor físico. Essa tecnologia oferece uma abordagem eficiente e escalável para virtualização, permitindo a execução de múltiplos sistemas operacionais em um único host.

\subsection{Eficiência e baixa sobrecarga}
 Ao contrário das soluções de virtualização completa, onde cada máquina virtual (VM) executa seu próprio kernel do sistema operacional, o OpenVZ compartilha um único kernel entre os contêineres. Essa abordagem compartilhada resulta em um melhor desempenho e eficiência, com menor utilização de recursos, uma vez que os contêineres não precisam carregar um sistema operacional completo.

\subsection{Isolamento avançado}
Cada contêiner é completamente isolado, com seu próprio sistema de arquivos, processos, rede e recursos alocados, proporcionando um ambiente seguro e protegido. O isolamento impede que os contêineres afetem uns aos outros e permite que cada contêiner opere independentemente, como se fosse uma instância virtualizada separada.

Além disso, o OpenVZ fornece recursos avançados de gerenciamento de recursos. Os recursos, como CPU, memória e espaço em disco, podem ser alocados e limitados de forma granular para cada contêiner. Isso permite um controle preciso e equilibrado do uso de recursos, garantindo que nenhum contêiner monopolize os recursos do sistema.

\subsection{Escalabilidade}
Sendo possível criar e gerenciar facilmente um grande número de contêineres em um único servidor físico, essa capacidade de dimensionamento horizontal permite que os administradores maximizem a utilização dos recursos e implantem aplicativos em escala, atendendo a demandas crescentes.

\subsection{Gerenciamento e monitoramento abrangentes}
 Criação trazendo  facilitação para uma administração dos contêineres, existindo interfaces de linha de comando e interfaces gráficas disponíveis para gerenciar, monitorar e fazer ajustes nos contêineres.
\section{Podman}

O Podman é uma ferramenta de gerenciamento de contêineres de código aberto que permite a criação, execução e gerenciamento de contêineres no ecossistema Linux. Ele é projetado para ser uma alternativa ao Docker, fornecendo recursos semelhantes enquanto se concentra em segurança, simplicidade e compatibilidade com o padrão Open Container Initiative (OCI).


\subsection{Daemon}
Ao contrário do Docker, que usa um daemon em execução em segundo plano, o Podman utiliza uma arquitetura sem daemon, executando os contêineres diretamente como processos no espaço de usuário. Isso resulta em um ambiente de execução mais seguro, eliminando a necessidade de um processo em execução em segundo plano com privilégios elevados.

\subsection{Isolamento robusto}
O Podman também oferece isolamento robusto de contêineres por meio do uso de namespaces e cgroups do kernel Linux. Ele permite a criação de contêineres isolados que possuem seu próprio ambiente de sistema de arquivos, processos, rede e outros recursos. Isso garante que os contêineres executados pelo Podman operem de forma independente, sem afetar uns aos outros ou o sistema operacional hospedeiro.

\subsection{Padrão OCI}
Significa que os contêineres criados com o Podman são compatíveis com outros sistemas de execução de contêineres que aderem ao mesmo padrão, como o Docker. Isso permite que os usuários compartilhem e distribuam contêineres facilmente entre diferentes ambientes e ferramentas compatíveis com OCI.

O Podman fornece um conjunto abrangente de comandos de linha de comando que permitem a criação, execução e gerenciamento de contêineres. Ele suporta recursos avançados, como criação de redes virtuais, montagem de volumes, configuração de variáveis de ambiente e execução de comandos dentro de contêineres em execução. Isso oferece flexibilidade e controle aos usuários para personalizar e ajustar seus contêineres de acordo com suas necessidades.

\subsection{Segurança}
 Podman oferece suporte a recursos de segurança, como o uso de SELinux para aplicar políticas de segurança e restrições adicionais nos contêineres. Ele também oferece recursos de gerenciamento de imagens e compartilhamento de contêineres através de registros e repositórios.
\section{Seccomp}

O Seccomp é uma ferramenta de segurança avançada disponível no kernel do Linux que permite restringir as chamadas de sistema (system calls) disponíveis para um processo em execução. Sua principal finalidade é aumentar a segurança do sistema, limitando as possíveis vias de ataque que um aplicativo pode explorar.

Uma das principais características do Seccomp é a capacidade de filtrar as chamadas de sistema permitidas para um processo. Ele permite que os administradores ou desenvolvedores definam uma política de segurança, especificando quais chamadas de sistema são permitidas e quais devem ser bloqueadas. Isso reduz a superfície de ataque do aplicativo, pois impede que ele faça chamadas de sistema potencialmente perigosas ou desnecessárias.

O Seccomp funciona usando um filtro BPF (Berkeley Packet Filter), que é um mecanismo de filtragem de pacotes flexível e eficiente. O filtro BPF é usado para especificar as regras do Seccomp, determinando quais chamadas de sistema são permitidas ou bloqueadas com base em critérios definidos. Essas regras podem ser aplicadas a um processo específico ou a um grupo de processos.

Outra característica importante do Seccomp é sua capacidade de fornecer um ambiente de execução seguro para aplicativos não confiáveis ou suspeitos. Por exemplo, em ambientes de contêineres, o Seccomp pode ser usado para restringir as chamadas de sistema disponíveis para um contêiner, garantindo que ele não acesse recursos ou execute operações não autorizadas. Isso ajuda a mitigar o impacto de possíveis ataques e limita as ações que um aplicativo malicioso pode realizar.

Além disso, o Seccomp é altamente flexível e suporta diferentes níveis de filtragem. Pode ser usado tanto em modo de auditoria, apenas registrando as chamadas de sistema que seriam bloqueadas, quanto em modo de execução real, bloqueando efetivamente as chamadas de sistema não permitidas. Isso permite que os administradores escolham o nível de restrição mais adequado para suas necessidades de segurança.

O Seccomp é amplamente utilizado em ambientes onde a segurança é uma prioridade, como contêineres, servidores web, sistemas embarcados e outros. Ao restringir as chamadas de sistema, ele ajuda a mitigar vulnerabilidades de segurança e limita as ações que um aplicativo comprometido pode realizar, aumentando a proteção do sistema como um todo.
 