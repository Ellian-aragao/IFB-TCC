\begin{tabular}{|c|p{5cm}|p{5cm}|}
\hline
\textbf{Parametro} & \textbf{Máquina Virtual} & \textbf{Container} \\
\hline
OS convidado    & Roda sobre hardware virtual e kernel é carregado em sua própria região de memória & Todos os convidados compartilham o mesmo kernel, e a imagem do kernel é carregado na memória física \\
\hline
Comunicação     & Será feita através de dispositivos de rede & Mecanismos padrões de IPC, como sinais, pipes, sockets, etc \\
\hline
Segurança       & Depende da implementação do Hypervisor & Mecanismos de controle do kernel \\
\hline
Performance     & Sofre do overhead das instruções a serem traduzidas do convidado para o host & Provém uma performance quase nativa \\
\hline
Isolamento      & Bibliotecas compartilhadas, arquivos e etc, não podem ser compartilhados entre convidado e host & Subdiretórios podem ser montados e compartilhados entre host e container \\
\hline
Tempo de inicio & Precisam de alguns minutos para iniciar  &  Podem ser iniciados em segundos \\
\hline
Armazenamento  & Precisa de muito armazenamento para seu kernel, programas e dependências &  Ocupam menos espaço uma vez que o kernel é compartilhado \\
\hline
\end{tabular}

\label{fig:container_vs_virt_t1_vs_virt_t2}