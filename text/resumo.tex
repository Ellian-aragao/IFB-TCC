A disponibilização de diversos Software acaba sendo algo muito comum na atualidade, ainda mais com larga adoção de internet de banda larga, em função disso, não só a disponibilidade de programas autênticos, temos também diversos malwares que acabam sendo acoplados a alguns softwares e distribuídos como se fossem a versão verídica destes. Por causa disto a infecção de computadores com malwares de sequestro de dados e encriptação de arquivos se tornaram tão comuns, a preocupação com a segurança vem tendo um crescente nas mais diversas frentes. Uma solução para tal é fazer a execução segura de softwares, conceito também conhecido como sandboxing, do qual uma aplicação é executada de forma isolada do sistema. Esta monografia faz um estudo sobre softwares que executam de forma segura outros programas, de modo a analisar as tecnologias bases aplicadas a eles, comparando e analisando questões como facilidade de uso e desempenho, além de propor casos de uso mais adequados quando aplicados dentro do sistema operacional Linux.

\begin{keywords}
Segurança de Software, Sandbox, Linux, Containers, Virtualização, Ferramentas
\end{keywords}