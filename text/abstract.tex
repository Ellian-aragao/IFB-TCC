The availability of various software ends up being something very common nowadays, even more with the wide adoption of broadband internet, because of this, not only the availability of authentic programs, we also have several malware that end up being coupled with some software and distributed as if they were the true version of these. Because of this the infection of computers with data hijacking and file encryption malware has become so common, the concern with security has been growing on the most diverse fronts. One solution is to make software execution secure, a concept also known as sandboxing, in which an application is executed in isolation from the system. This monograph makes a study of software that securely executes other programs, in order to analyze the base technologies applied to them, comparing and analyzing issues such as user friendly and performance, as well as proposing more appropriate use cases when applied within the Linux operating system.


\begin{keywords}
Software Security, Sandbox, Linux, Containers, Virtualization, Tools
\end{keywords}