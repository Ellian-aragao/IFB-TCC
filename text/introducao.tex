\chapter{Introdução}
\label{chp:introduction}

Conforme o uso de software para as mais variadas necessidades, torna-se cada vez mais comum a execução de programas, seja para atividades triviais ou para solucionar problemas complexos, de todo modo sendo cada vez mais parte do cotidiano da sociedade, uma vez que facilita processos das mais diversos tipos \cite{bctv}.
 
Entretanto, a exploração das mais variadas vulnerabilidades nos sistemas para diversas questões, seja a exploração de informações como Ransomware, que consiste em fazer encriptação de todos os dados e pedir dinheiro em troca da desencriptação dos dados, e também o ataques de Supply Chain, do qual se ataca a cadeia de suprimento do software, como atacar um software base de uma outra solução para obter seus dados. Assim tendo riscos contantes frente ao mundo o qual processos são geridos por sistemas, casos como o ransoware no ministério da saúde e o caso da Solar Wings, que popularizou ataques a sistemas por meio do Supply Chain. Sendo também observado no ecossistema Open source, encontraram um pacote do NPM, gerenciador de pacotes do ecossistema javascript, o qual utilizava a máquina do desenvolvedor para mineração de criptomoedas \cite{Magalhães_Filho_Marcheri_2022, 10.1007/978-981-13-1274-8_31, 9799263, greenberg2017supply, cpomagazine, aquasec}.

Frente estas questões a execução protegida de programas tem sido um tema cada vez mais importante, por garantir a segurança para execução de aplicações sem o constante receio de ser atacado, assim conceitos como virtualização, conteinerização e sandboxing tem se tornado opções para resolução de tais questões, aliadas as ferramentas que utilizam destas bases para resolver o problema de uma forma amigável ao usuário \cite{10.1145/2884781.2884782, 10.1145/3344341.3368810}.

Sobre as tecnologias citadas, temos a virtualização da qual será o primeiro passo para as tecnologias, esta consiste em uma representação virtual dos recursos, sejam um workspace isolado do qual um software quando executado não possui acesso direto os recursos que o kernel provêm. Nestas formas temos a virtualização utilizando uma máquina virtual (MV), do qual rodamos um sistema operacional e este intervem nas chamadas, o que torna mais pesada a virtualização, quando em comparação a containerização. Os containers por sua vez utilizam de recursos do kernel que são os namespaces, que fazem o isolamento do acesso a recursos, e o controle de grupos (cgroups), o qual limita o uso de recursos de entrada e saída do sistema operacional, desta forma o processo roda ainda no SO diretamente mas com as limitações impostas pelos namespaces e cgroups, sendo tão rápido quanto inicialização de um software. Desta forma representando a principal diferença entre os tipos de virtualização \cite{kernelscheepers, secrypt22}.

Compreendida a virtualização, temos então as ferramentas que facilitaram o uso destes recursos, em que cada uma dado suas diferenças serão comparadas para determinar seus casos de uso mais adequados e compreendido suas integrações com demais ferramentas para construção de um ambiente de execução mais seguro para teste das aplicações e comportamentos que esta pode vir a ter.

Por fim, esta monografia visa apresentar todo o referencial teórico do funcionamento das ferramentas, e a partir disto apresentar estas no capítulo de ferramentas e através da metodologia, diferenciar e comparar cada uma destas para conclusão de casos de uso mais adequados e aplicáveis a cada uma, tendo assim apresentado uma revisão sistemática de seus fundamentos e aplicação de modo a auxiliar a escolha de cada uma para sua respectiva função de modo reprodutível e direto ao ponto, tal como instruções da aplicabilidade de cada caso de uso nestas. Tendo exemplificado a proposta a seguir nos objetivos desta monografia.

\section*{Objetivo Geral}
Este trabalho visa compreender quais são as bases das ferramentas de sandbox aplicadas ao Linux, e através desta compreensão fazer análises de segurança, performance e usabilidades destas para definir quais seriam os casos de uso adequados a cada uma destas.

\section*{Objetivos Específicos}
\begin{itemize}
    \item Fornecer uma lista de ferramentas de sandbox;
    \item Fornecer um guia de uso das ferramentas;
    \item Compreender diferenças entre as ferramentas;
    \item Compreender caso de uso de cada ferramenta.
\end{itemize}

