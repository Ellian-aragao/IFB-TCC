\chapter{Introdução}
\label{chp:introduction}

Conforme o uso de software para as mais variadas necessidades, torna-se cada vez mais comum a execução de programas, seja para atividades triviais ou para solucionar problemas complexos, de todo modo sendo cada vez mais parte do cotidiano da sociedade, uma vez que facilita processos das mais diversos tipos \cite{bctv}.
 
Entretanto, em função do crescimento da adoção, tem se tornado cada vez mais comum diversos golpes e exploração das mais variadas vulnerabilidades nos sistemas para diversas questões, seja a exploração de informações,  sequestro de dados (Ransomware) e ataques de Supply Chain. Assim tendo riscos contantes frente a o mundo conectado via Internet e do qual os processos são geridos por sistemas, casos como o do ministério da saúde acabam sendo comuns, outro caso emblemático é da Solar Wings, que popularizou ataques a sistemas por meio do Supply Chain, caso tal que também é observado no ecossistema Open source, um pacote do NPM (ecossistema javascript) do qual utilizava a máquina do desenvolvedor para mineração de criptomoedas \cite{Magalhães_Filho_Marcheri_2022, 10.1007/978-981-13-1274-8_31, 9799263, greenberg2017supply, cpomagazine, aquasec}.

Frente estas questões a execução protegida de programas tem sido um tema cada vez mais importante, por garantir a segurança para execução de aplicações sem o constante receio de ser atacado, assim conceitos como virtualização, conteinerização e sandboxing tem se tornado opções para resolução de tais questões, tanto quanto a análise dos componentes sendo executados. \cite{10.1145/2884781.2884782, 10.1145/3344341.3368810}.